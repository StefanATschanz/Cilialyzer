\documentclass{article}
\usepackage{graphicx}


\usepackage{hyperref}
\usepackage[utf8]{inputenc}
\usepackage{geometry}
\geometry{margin=1in}

\begin{document}

\title{Cilialyzer Notes}
\author{Martin Schneiter}

\maketitle

\section{Running The Cilialyzer-Software From The Python Interpreter}
\subsection{Install The Latest Python 3 Release}
If you wish to run the Cilialyzer-software from the python interpreter, 
you have to install Python 3. 
The latest release of Python 3 can be downloaded from the 
official website \url{www.python.org}
(where you need to choose the right MSI installer for Windows: 
most probably ``Windows x86-64 MSI installer''). 
\subsection{Install the Required Packages}
After having installed the core of Python 3.XX, you are now able to 
run Python's package manager, which is named ``pip''
(recursive acronym: ``pip installs packages''). 
Pip enables you to convienently install (and manage) any Python compliant 
packages.\par
In order to being able to run the Cilialyzer software, 
you will need to install the following list of packages for scientific 
computing: \textbf{pillow, numpy, matplotlib, scipy}. 
All these packages can be installed with the same pip-command 
(replacing the respective package-name): 
\textit{pip install package-name}~.


\section{Planned Next Steps}

\begin{itemize}

\item Process bars $\rightarrow$ evtl. anpassen, da diese durch Rückmeldung ans Frontend 
	den jeweiligen Prozess verlangsamen 
	 
\item 100\% Zoom initial 

\item Powerspec $\rightarrow$ beim neusetzen der Grenzen, CBF nicht überschreiben

\item drehbarer ROI, bzw. drehbare Sequenz 
	 
\item \textbf{Extraction of The Ciliary Motion}, 
Preprocessing (Image Registration / Image Alignment / Image Stabilization) 
$\rightarrow$ ähnlich zu Puybareau et al. 2016: 
Wegrechnen der Vibrationen UND Zellbewegung. Nur die Zilien sollen sich bewegen! 
In E. Puybareau wird SIFT, SURF vorgeschlagen. Diese sind allerdings patentiert. 
ORB wäre eine freie alternative zu SIFT/SURF. Anschliessend könnte die gesuchte 
Rotation und Translation mit einem Ausgleichsverfahren (bspw. RANSAC) bestimmt werden. 
Evtl. einfacher: Via Phase correlation. Dazu ist ein python-Paket vorhanden: "pystackreg". 

\item Vorschlag eines "nächsten Ordners". Entweder ein Vorschlag alphanumerisch. Oder evtl. 
mit Scroll-Auswahl (Sub-Directories auf derselben Hierarchie-Ebene) 

\item Bemerkung: Durch das Entfernen vom 'animationtab' wird 'Flipbook.py' 
nicht mehr verwendet. Könnte man löschen?  

\item i think that plenty of code can be removed in cilialyzer.py: 
	  object 'player' seems unnecessary? delete animate\_roi? 


\end{itemize} 








\end{document}
